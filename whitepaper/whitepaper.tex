\documentclass[12pt]{article}
\usepackage[utf8]{inputenc}
\usepackage{graphicx}
\usepackage{amsmath}
\usepackage{hyperref}
\usepackage{amsfonts}
\usepackage{amssymb}
\usepackage{graphicx}
\usepackage{subcaption}
\usepackage{float}
\usepackage{listings}
\usepackage{color}
\usepackage{parskip}
\usepackage[left=1in,right=1in,top=1in,bottom=1in]{geometry}

\title{Rotacrypt: Rotational Mechanics as a Cryptographic Primitive}
\author{Teo Honda Scully}
\date{}

\begin{document}

\maketitle

\begin{abstract}
...TBD...
\end{abstract}

\tableofcontents

\newpage

\section{Introduction}
<...>

\subsection{Review of Rubik's Cubes}
Ah, the Rubik's Cube—the iconic toy that bedevils and delights in equal measure. Born from the ingenious mind of Ernő Rubik in 1974, it's more than just a puzzle. Forget mere child's play; this cube is chaos in the physical. Just kidding. To go from the state of chaos to order, one only needs to know a solving protocol of which many exist.

Max Park, the current world record holder (11 June 2023) for the 2-handed solve, obliterated the cube in a mind-blowing 3.13 seconds. Lucky scramble? Hardly. Yiheng Wang, who holds the world record average-of-five, clocks in at a dizzying 4.48 second mean solve time throughout the five solves \footnote{An average-of-five is determined by taking the average of the three "middle" solves in a session of five scramble. In other words, the worst and best solve time are dropped from the calculation for the average.}. And no, I won't depress you by mentioning his tender age of nine years old. But let's not divert. The Rubik's Cube--a mathematical marvel and a cipher waiting to be cracked. 

Let's dive into the details of the mechanics of the 3x3x3 puzzle. The cube boasts centers, edges, and corners. You can spin its layers, but let's be clear: it's not as simple as it looks. When you rotate one layer, the whole cube changes. Confused? That's the point. This cube was designed to keep you on your toes, to challenge your notions of space and symmetry.

\begin{figure}[H]
    \hspace{20pt}
    \begin{minipage}{.3\textwidth}
        \includegraphics[scale=0.1]{cube.png}
    \end{minipage}%
    \begin{minipage}{.6\textwidth}
        A visualization of a 3x3x3 cube. The cube has 6 faces, each with 9 stickers. Notably, this means that there exists 54 different unit tiles on the cube. The cube has 43,252,003,274,489,856,000 possible states.
    \end{minipage}
    
    \vspace{20pt}
    
    \hspace{20pt}
    \begin{minipage}{.3\textwidth}
        \includegraphics[scale=0.1]{core.png}
    \end{minipage}%
    \begin{minipage}{.6\textwidth}
        The six colors of the cube are Yellow, Blue, Red, Green, Orange, and White. The cube has 6 center pieces, and they are permanently fixed in place. The center pieces are the only pieces that have a single color, and they do not move when the cube is being solved.\\
    \end{minipage}
\end{figure}

The cube is constructed of six faces, each of which has a central piece permanently affixed to the core. These center pieces serve as the invariant axis around which the peripheral cubies rotate. In each face, surrounding the central piece, are four edge pieces with two-colored stickers and four corner pieces with three-colored stickers.\\

\subsection{Scheme Overview}
<...>

\subsection{Intended Usage}
<...>

\section{3x3x3 Implementation}
<...>

\subsection{Data Structure Breakdown}
<...>

\subsection{Augmented SPEFFZ Mapping}
<...>

\subsection{Cyclic Transformations}

\section{Key Generation}
<...>

\subsection{4-Cube Initialization}
<...>

\subsection{Master-Key Serialization}
<...>

\subsection{Sub-Key Generation}
<...>

\section{Encryption}
<...>

\subsection{Plaintext Setup With S-Box Transformations On Chunks}
<...>

\subsection{Cube Mapping Procedure}
<...>

\subsection{Encryption Algorithm}
<...>

\section{Decryption}
<...>

\end{document}
